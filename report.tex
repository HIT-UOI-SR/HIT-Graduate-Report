% !TEX encoding = UTF-8
\documentclass{hitgsrep}
\usepackage{graphicx}
\usepackage{amsmath}
\usepackage{physics}
\usepackage[outputdir={out}]{minted}% 代码高亮,需要Pygments

\hitgsrepset{
    author={张三}, % 学生姓名
    studentid={19SXXXXXX}, % 学号
    studentcat={工学硕士}, % 学生类别
    course={数值分析}, % 考核科目
    faculty={仪器科学与工程学院}, % 学生所在院(系)
    discipline={仪器科学与技术}, % 学生所在学科
%    year={\the\year}, % 年份(不填根据当前时间自动生成)
%    semester={秋}, % 学期(不填根据当前时间自动生成)
}

\newminted[wlcode]{wl}{linenos,autogobble,style=mathematica}% Mathemtica高亮,需要pygments-mathematica

\begin{document}
\maketitle

\section{龙贝格积分法}

\begin{abstract}
    对于实际的工程积分问题,很难应用-莱布尼兹公式去求解。
    因此应用数值方法进行求解积分问题已经有着很广泛的应用,本文基于龙贝格积分法来解决一类积分问题。
\end{abstract}

\subsection{数学原理}

考虑积分$I(f)=\int_a^b f(x)\dd{x}$,欲求其近似值,通常有复化的梯形公式、辛普森公式和科特斯公式。
但是给定一个精度,这些公式达到要求的速度很缓慢。如何提高收敛速度,自然是人们极为关心的课题。
为此,记$T_{1,k}$为将区间$[a,b]$进行$2^k$等分的复化的梯形公式计算结果,
相仿的,记$T_{2,k}$为将区间$[a,b]$进行$2^k$等分的复化的辛普森公式计算结果,等等。
根据Richardson外推加速方法,可以得到收敛速度较快的龙贝格积分法。其具体的计算公式为:
\begin{enumerate}
    \item 准备初值,计算
    $$
        T_{1,1}=\frac{b-a}{2}[f(a)+f(b)]
    $$
    \item\label{itm:rec} 按梯形公式的递推关系,计算
    $$
        T_{1,k+1}=\frac{1}{2}T_{1,k}+\frac{b-a}{2^k}\sum_{i=0}^{2^{k-1}-1}f\qty(a+\frac{b-a}{2^{k-1}}(i+0.5))
    $$
    \item 按龙贝格积分公式计算加速值
    $$
        T_{m,k-m}=\frac{4^{m-1}T_{m-1,k+1-m}-T_{m-1,k-m}}{4^{m-1}-1}
    $$
    \item 精度控制。对给定的精度$R$,若
    $$
        \abs{T_{m,1}-T_{m-1,1}}<R
    $$
    则终止计算,并取$T_{m,1}$为最终结果;否则返回 \ref{itm:rec} 重复计算,直到满足要求的精度为止。
\end{enumerate}

\subsection{程序设计}

龙贝格积分函数:
\begin{wlcode}
romberg[f_, {a_, b_}] := Module[
  {k = 1,
   h = b-a,
   n = 1,
   t = {((b-a)/2.)*(f[a]+f[b])}
  },
  While[
    AppendTo[t,
      t[[-1]]/2+(h/2)*Sum[f[a+h*(i+1/2)],{i,0,n-1}]
    ];
    Do[
      t[[k-m+1]] = (4^m*t[[k-m+2]]-t[[k-m+1]])/(4^m-1),
      {m, 1, k}
    ];
    h /= 2;
    n *= 2;
    ++k;
    t[[1]] != t[[2]]
  ];
  t[[1]]
]
\end{wlcode}

\end{document}
